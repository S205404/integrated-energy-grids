\documentclass[10pt]{article}
\usepackage{graphicx} % Required for inserting images
\usepackage{url}
\usepackage{hyperref}
\title{IEG_Problems_Lecture8}
\author{martavictoriaperez }
\date{April 2025}

\usepackage[margin=1in]{geometry} 
\usepackage{amsmath,amsthm,amssymb, graphicx, multicol, array}
 
\newcommand{\N}{\mathbb{N}}
\newcommand{\Z}{\mathbb{Z}}
 
\newenvironment{problem}[2][Problem]{\begin{trivlist}
\item[\hskip \labelsep {\bfseries #1}\hskip \labelsep {\bfseries #2.}]}{\end{trivlist}}

\begin{document}
 
\title{\textbf{Lecture 12: Multi-carrier energy systems II: aviation, shipping, industry}}
\author{DTU Course 46770: Integrated Energy Grids }
\maketitle

\begin{problem}{12.1}

In this problem, we want to build a stylized mode of hydrogen production in an energy island. Assume an offshore generator and an electrolyzer with the cost in Table \ref{tab:my_label}. For the offshore generator, assume the capacity factors for Denmark in \url{https://zenodo.org/record/3253876#.XSiVOEdS8l0}.

\

An annual hydrogen demand of 200 GWh must be delivered and, for the sake of simplicity, assume that the island includes a hydrogen storage with no cost. The electrolyzer efficiency is assumed to be 62\%.



\begin{table}[h]
    \centering
    \begin{tabular}{ccc}
    \hline
        Technology & Annualized capital costs (EUR/MW/a) & Marginal generation costs (EUR/MWh) \\
    \hline
    Offshore Wind &  174,556 & 0 \\
         Electrolyzer & 188,715 & 0 \\
    \hline
    \end{tabular}
    \caption{Costs assumptions.}
    \label{tab:my_label}
\end{table}

\begin{itemize}
\item[a)] What is the optimum capacity of offshore wind and electrolyzer that needs to be installed?

\item[b)] What is the optimal storage capacity of hydrogen, in absolute terms and relative to the annual demand?

\item[c)] Plot the duration curve for offshore wind generation and electrolyzer operation and discuss the results.

\item[d)] At what cost can the H$_2$ be produced and how does it compare to current H$_2$ price?

\end{itemize}


\end{problem}

\

\begin{problem}{12.2}

In this problem, we want to build a stylized mode of methanol production in an energy island. Assume an offshore generator, an electrolyzer, a Direct Air Capture (DAC) unit ,and a methanolisation unit with the cost in Table \ref{tab_cost}. For the offshore generator, assume the capacity factors for Denmark in \url{https://zenodo.org/record/3253876#.XSiVOEdS8l0}.

\

An annual methanol demand of 200 GWh must be delivered and, for the sake of simplicity, assume that the island includes a hydrogen storage, a CO$_2$, and a methanol storage with no cost. The electrolyzer efficiency is assumed to be 62\%. The methanolisation plant requires hydrogen, CO$_2$ and electricity as inputs. It produces 0.8787 MWh of methanol per MWh of hydrogen, 4.0321 MWh of methanol per tonne of CO$_2$, and 3.6907 MWh of methanol per MWh of electricity. 

\

The DAC unit requires  0.55 MWh of electricity and 1.4 MWh of heat to capture 1 tonne of CO$_2$. Heat is assumed to be provided by a heat pump with a constant coefficient of performance (COP) of 3.



\begin{table}[h]
    \centering
    \begin{tabular}{ccc}
    \hline
        Technology & Annualized capital costs (EUR/MW/a) & Marginal generation costs (EUR/MWh) \\
    \hline
    Offshore Wind &  174,556 & 0 \\
    Electrolyzer & 188,715 & 0 \\
    Methanolisation & 87,538 & 0 \\
    Direct Air Capture (DAC) & 863,357 & 0 \\
    \hline
    \end{tabular}
    \caption{Costs assumptions.}
    \label{tab_cost}
\end{table}

\begin{itemize}
\item[a)] What is the optimum capacity of offshore wind, electrolyzer, DAC, and methanolisation that needs to be installed?
\item[b)] What is the optimal storage capacity of hydrogen, CO$_2$, and methanol, in absolute terms and relative to the annual demand?
\item[c)] Plot the duration curve for the offshore wind generation, electrolyzer, DAC and methanolisation operation and discuss the results.
\item[d)] At what cost can the methanol be produced and how does it compare to current methanol price?

\end{itemize}


\end{problem}



%\begin{proof}[Solution]
%Write a solution here
%\end{proof}

\end{document}


 

