\documentclass[10pt]{article}
\usepackage{graphicx} % Required for inserting images
\usepackage{url}
\usepackage{hyperref}
\title{IEG_Problems_Lecture_10}
\author{Aleksander Grochowic}
\date{March 2025}

\usepackage[margin=1in]{geometry} 
\usepackage{amsmath,amsthm,amssymb, graphicx, multicol, array}
 
\newcommand{\N}{\mathbb{N}}
\newcommand{\Z}{\mathbb{Z}}
 
\newenvironment{problem}[2][Problem]{\begin{trivlist}
\item[\hskip \labelsep {\bfseries #1}\hskip \labelsep {\bfseries #2.}]}{\end{trivlist}}

\begin{document}
 
\title{\textbf{Lecture 10: Capacity and dispatch optimisation in a network}}
\author{
%Your name\\
DTU Course 46770: Integrated Energy Grids }
\maketitle

\begin{problem}{10.1}
We build on the models described in Problem sets 8 and 9:

Optimize the capacity and dispatch of solar PV, onshore wind, and Open Cycle Gas Turbine (OCGT) generators to supply the inelastic electricity demand throughout one year. 
To do this, take the time series for the wind and solar capacity factors for Portugal and Denmark in 2015 obtained from \url{https://zenodo.org/record/3253876#.XSiVOEdS8l0}
and \url{https://zenodo.org/record/2613651#.X0kbhDVS-uV} (select the file ‘pvoptimal.csv’) and the electricity demand from \url{https://github.com/martavp/integrated-energy-grids/tree/main/integrated-energy-grids/Problems/data}. Note that we work without a $CO_2$ constraint in this exercise.

\

Consider the annualized capital costs and marginal generation costs for the different technologies in the following table. The efficiency for the OCGT plant is 0.41.

\begin{table}[h]
	\centering
	\begin{tabular}{cccc}
		\hline
		Technology & Capital costs (EUR/MW/a) & Marginal costs (EUR/MWh) & $CO_2$ emissions (t$CO_2$/MWh_{th}) \\
		\hline
		Onshore Wind &  101,644 & 0 & 0 \\
		Solar PV &  51,346 & 0 & 0 \\
		OCGT & 47,718 &  64.7 & 0.198  \\
		\hline
	\end{tabular}
	\caption{Costs assumptions.}
	\label{tab:my_label}
\end{table}


Now, however, we add Denmark as a second node and we connect Portugal and Denmark with an overhead HVAC line. Assume that the annualised capital cost of it is 42 EUR/MWkm/a and that the distance is 2477 km. Set up a network with two nodes and connect them with an overhead AC line.

For (a)-(d), calculate the following:
\begin{itemize}
	\item total system costs (in bn EUR)
	\item average electricity price (in EUR/MWh) and number/share of hours with zero prices.
	\item congestion rent 
	\item utilisation of transmission lines (in \% of capacity)
	\item total generation per technology (in TWh)
	\item total $CO_2$ emissions (in Mt$CO_2$)
\end{itemize}

\begin{itemize}
	\item[a)] Solve the network in PyPSA while keeping the capacity at 0 GW and assuming it cannot be extended.
	\item[b)] Now assume the AC line connecting Portugal and Denmark has a capacity of 1 GW.
	\item[c)] Now assume the AC line connecting Portugal and Denmark has a capacity of 10 GW.
	\item[d)] Optimise the AC line capacity endogenously (assume its current capacity is 0 GW). Are the costs recovered by the congestion rent?
\end{itemize}


\end{problem}

\

\begin{problem}{10.2}
We build on Problem 10.1d), i.e. we optimise the capacity of the transmission line between Denmark and Portugal endogenously (starting at 0 GW). For (a)-(c) calculate the following:
\begin{itemize}
	\item total system costs (in bn EUR)
	\item average electricity price (in EUR/MWh) and number/share of hours with zero prices.
	\item congestion rent 
	\item utilisation of transmission lines (in \% of capacity)
	\item total generation per technology (in TWh)
	\item $CO_2$ price (EUR/t$CO_2$)
\end{itemize}




\begin{itemize}
	\item[a)] Set up a network as in 10.1d) and add a CO2 constraint of 2.5 MtCO2/year.
	\item[b)] The load distribution between Denmark and Portugal is approximately 1:1.5. Impose national targets  Assume that both countries have national targets of 1 MtCO2/year and 1.5 MtCO2/year, respectively.
	\item[c)] Assume the CO2 limits from b), but that the transmission capacities is limited to 1 GW (still endougenously optimised).
	\item[d)] Change the global CO2 limit to 0.25, 0.5, 1, 1.5, 2 MtCO2/year, and plot the total system costs against the CO2 limit.
	
\end{itemize}

\textbf{Hint}: \\
In order to add national targets for carbon emissions, you need to manually add a constraint and modify the model formulation. 

For this, you first need to access the underlying model in linopy with \texttt{n.optimize.create\_model()}. 
Then you add the constraint with \texttt{n.model.add\_constraint(lhs, "<=", rhs, name)}. One way to achieve this is to access the aggregated operational variables for the generators (solar, wind, OCGT) with \texttt{gen\_var = n.model.variables["Generator-p"].sum("snapshot")}. Then you multiply this with the corresponding $CO_2$ emission factor for each generator, and group the corresponding constraints as follows: \texttt{lhs = (gen\_var * co2\_emission\_factors).groupby(n.generators.bus.to\_xarray()).sum()}. In order to now optimise the modified model you need to run \texttt{n.optimize.solve\_model()}. You can access the dual variables at the end with \texttt{n.model.dual()}.



\end{problem}

\




%\begin{proof}[Solution]
%Write a solution here
%\end{proof}

\end{document}


 

