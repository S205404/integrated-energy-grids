\documentclass[10pt]{article}
\usepackage{graphicx} % Required for inserting images
\usepackage{url}
\usepackage{hyperref}
\title{IEG_Problems_Lecture8}
\author{martavictoriaperez }
\date{March 2025}

\usepackage[margin=1in]{geometry} 
\usepackage{amsmath,amsthm,amssymb, graphicx, multicol, array}
 
\newcommand{\N}{\mathbb{N}}
\newcommand{\Z}{\mathbb{Z}}
 
\newenvironment{problem}[2][Problem]{\begin{trivlist}
\item[\hskip \labelsep {\bfseries #1}\hskip \labelsep {\bfseries #2.}]}{\end{trivlist}}

\begin{document}
 
\title{\textbf{Lecture 9: Limiting CO2 emissions}}
\author{
%Your name\\
DTU Course 46770: Integrated Energy Grids }
\maketitle

\begin{problem}{9.1}
Use the model described in Problem 8.2 and assume that methane gas emits 0.198 tCO2 per MWh of thermal energy contained in the gas. 

Consider the annualized capital costs and marginal generation costs for the different technologies in the following table.

\begin{table}[h]
    \centering
    \begin{tabular}{ccc}
    \hline
        Technology & Annualized capital costs (EUR/MW/a) & Marginal generation costs (EUR/MWh) \\
    \hline
    Onshower Wind &  101,644 & 0 \\
         Solar PV &  51,346 & 0 \\
         OCGT & 47,718 &  64.7  \\
         CCGT & 104,788 &  46.8   \\
         Battery inverter & 12,894  & 0 \\
         Battery energy capacity &  24,678 & 0 \\
    \hline
    \end{tabular}
    \caption{Costs assumptions.}
    \label{tab:my_label}
\end{table}


Limit the maximum CO2 emissions to 5 MtCO2/year. 

\begin{itemize}
\item[a)] Calculate the optimal installed capacities and plot the hourly generation and demand during January.
\item[b)] What is the CO2 tax required to attain such CO2 emissions limit?
%\item[c)] Calculate the revenues collected by the OCGT plant throughout the year and show that their sum is equal to its costs.
%\item[d)] Repeat the previous sections assuming now a zero-emission constraint.
\end{itemize}

\end{problem}


%\begin{proof}[Solution]
%Write a solution here
%\end{proof}

\end{document}


 

